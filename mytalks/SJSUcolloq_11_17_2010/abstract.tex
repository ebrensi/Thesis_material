\documentclass[]{article}
\newcommand{\krylov}[3]{\mathcal{K}_{#1}({#2},{#3})}

\title{Krylov Subspaces and Their Application to Model Order Reduction}

\begin{document}
\maketitle
\begin{abstract}
Given a matrix $H$ and vector $r$, the n-th Krylov subspace $\krylov{n}{H}{r}$ is the space spanned 
by the vectors $r, Hr,  H^2r, \ldots, H^{n-1}r$.   Krylov subspaces are used extensively in numerical 
linear algebra, but computing a basis for one is non trivial.  I will explain why explicitly computing 
$Hr$, $H^2r$, $H^3r$ is not numerically feasible, and discuss a few strategies for 
generating a basis for a Krylov subspace.  Finally, I will talk about how Krylov subspaces are used in 
my field of research, Model Order Reduction.
\end{abstract}

\begin{itemize}
\item I'll assume the audience is familiar with some basic concepts in Linear Algebra, such as eigenvectors, and the notion of a subspace spanned by a set of vectors.  Also, to understand model reduction it would help to have familiarity with differential equations and Taylor series approximation.

\item Efrem Rensi is a 5th year Ph.D. student in Applied Math at UC Davis. He graduated from SJSU with a B.S. in Applied Math, in 2006.

\end{itemize}

\end{document}