\documentclass[serif]{beamer}

%\newcommand{\putfig}[3][{560 420}]{\includegraphics[bb=0 0 #1, width=#2\textwidth]{./talkfigs/#3}} % for DVI (latex)
%%\newcommand{\putfig}[3][{}]{\includegraphics[width=#2\textwidth]{./talkfigs/#3}} % for PDF (pdflatex)
\newcommand{\Balert}[1]{\textcolor{blue}{#1}}


\mode<presentation>
{   \usetheme{Copenhagen}
	%\usecolortheme{beaver}
}

\usepackage{bm,verbatim,graphicx,datetime,relsize}
\usepackage{color,braket}
\usepackage[english]{babel}
\usepackage[latin1]{inputenc}
\usepackage{times}
\usepackage[T1]{fontenc}
% Or whatever. Note that the encoding and the font should match. If T1
% does not look nice, try deleting the line with the fontenc.

%% ----------------------
\input ../../documentation/doc_macros

\usdate  % format for \today
\settimeformat{ampmtime}
%%% ---------------


\title[PCC-Krylov (Grad Seminar)] % (optional, use only with long paper titles)
{Realification and PCC-Krylov Subspaces for Projection-Based Model Reduction}

%\subtitle{} % (optional)

\author[]{Efrem Rensi}
\institute{UC Davis Mathematics (GGAM)}
% If you wish to uncover everything in a step-wise fashion, uncomment
% the following command: 
%\beamerdefaultoverlayspecification{<+->}

% *****************************************************
\begin{document}

\begin{frame}
  \titlepage
% \begin{figure}
%  	    \centering
%  	    \putfig[{324 149}]{0.30}{IO.png}
%  \end{figure}
\end{frame}



%\begin{comment}
\begin{frame}{Thanks  in Advance to Pizza Seminar Attendees!}
 \begin{figure}
  	    \centering
  	    \putfig[{399 314}]{0.75}{pizzabama.jpg}
  \end{figure}
\end{frame}

%\end{comment}

\begin{frame}{Unreduced Model}
  % - A title should summarize the slide in an understandable fashion
  %   for anyone how does not follow everything on the slide itself. 
   \emph{Input-Output} system represented as a system of 
   Differential Algebraic Equations (DAEs)
   \begin{equation*}  
   \begin{array}{c}
   u_1(t)\quad\longrightarrow \\ u_2(t)\quad\longrightarrow\\ \vdots\\u_\nin(t)\quad\longrightarrow
   \end{array}
  \quad
  \boxed{\large\Balert{%
 	    \begin{aligned}
 	    \\
 			\quad E x' &=  Ax + Bu\quad\\
 			y &= B^Tx,\\
 			\\
 		\end{aligned}
		}
      }
  \quad	
		 \begin{array}{c}
	 \longrightarrow\quad y_1(t)\\ \longrightarrow\quad y_2(t)\\ \vdots\\\longrightarrow\quad y_\nin(t)
   \end{array} 
 \end{equation*}
 where $A,E\in\R^{N\times N}$ are sparse (possibly singular), $B\in\R^{N\times \nin}$.
 
 \begin{itemize}
   \item $N \gg \nin$ typically large, e.g. $N=\bigO{10^{9}}$
   \item $x(t)\in\R^N$  represents internal state.
  % \item \emph{Behavior} of model: $y=F(u)$
   \end{itemize}
 \end{frame}


\begin{frame}{Reduced Order Model (ROM) via Projection}
   System of DAEs of the same form
   \begin{equation*}  
   \begin{array}{c}
   u_1(t)\quad\longrightarrow \\ u_2(t)\quad\longrightarrow\\ \vdots\\u_\nin(t)\quad\longrightarrow
   \end{array}
  \quad
  \boxed{\large\Balert{%
 	    \begin{aligned}
 	    \\
 			\quad \En x' &=  \An x + \Bn u\quad\\
 			y &= \Bn^Tx\\
 			\\
 		\end{aligned}	
      }
  	}
  \quad	
		 \begin{array}{c}
	 \longrightarrow\quad y_1(t)\\ \longrightarrow\quad y_2(t)\\ \vdots\\\longrightarrow\quad y_\nin(t)
   \end{array} 
 \end{equation*}
 
 \[
 	\An := V^TAV, \quad \En := V^TEV \quad\in \R^{n\times n}
 \]
 
 \[
 	\Bn := V^T\Bn \in\R^{n\times \nin},
 \]
with state-space dimension $n \ll N$ and $V\in\R^{N\times n}$ is
basis for some ideal space.  
 
\end{frame}



\begin{frame}{Reduced Model Must Preserve Original Structure}
Example: RCL circuit
\begin{figure}
 	    \centering
  	    \putfig[{723 543}]{0.70}{ttt.png}
\end{figure}
\end{frame}



\begin{frame}{Transfer Function}{Relates Output directly to Input in Frequency Domain}
Original system:
  \begin{align*}
   	E x' &=  Ax + Bu\\
   	 y &= B^Tx.
 \end{align*}
 Applying the Laplace transform,
  \begin{align*}
      sEX &=  AX(s) + BU(s)\\
      Y(s) &= B^TX(s).
 \end{align*}

In the frequency domain,
\[
Y(s) = \Balert{B^T(sE-A)^{-1}B} U(s) \equiv \Balert{H(s)}U(s).
\]
\end{frame}



\begin{frame}{Transfer Function}{Input $\rightarrow$ Output Map in Frequency Domain}
    In the frequency domain, $Y(s) = H(s)U(s)$ with transfer~function 
 	\begin{align*}
 		H(s) &=  B^T(sE-A)^{-1}B \quad\in\quad (\C\cup\infty)^{\nin \times \nin}
 	\end{align*}
 	For the reduced model,
 	\begin{align*}
	 	\widetilde{H}(s) &=  \Bn^T(s\En-\An)^{-1}\Bn \quad\in\quad (\C\cup\infty)^{\nin \times \nin}
 	\end{align*}
 	
 	\[
 		\widetilde{H}(s)\approx H(s) \quad\Longleftrightarrow\quad
 		\text{`Good' Reduced Order Model}
  	\]
\end{frame}
 


\begin{frame}{Moment Matching}
 	Expressed as Taylor series expansion about $\xp\in\C$:
 	\begin{align*}
 		\text{Original:}\quad H(s) &= \sum_{j=0}^\infty (s-\xp )^jM_j \\
 		\text{ROM:}\quad\widetilde{H}(s) &= \sum_{j=0}^\infty (s-\xp )^j\widetilde{M}_j\\
 	\end{align*}
 ROM matches $n$ moments about $\xp$ if
 $\widetilde{M}_j=M_j$ for $j=1,2,\ldots,n$.
\begin{itemize}
	\item Pad\`e-Type Approximant 
	\[ \widetilde{H}(s) =  H(s) + \bigO{(s-\xp )^{n}} \] 
\end{itemize}
\end{frame}






\begin{frame}{Transfer function}{Single-matrix formulation}
Choose expansion point $\xp\in\C$, re-write $H(s)$ as  
	\begin{align*}
			H(s) &= B^T(sE-A)^{-1}B\\
			    &= B^T\left[(s-\xp)E +\xp E- A\right]^{-1}B\\
				&= \alert{B^T\left(I-(s-\xp)\A\right)^{-1}R} 
	\end{align*}
(\alert{\emph{Single matrix formulation}}), where 
\[ \A := -(\xp E-A)^{-1}E\quad\text{and}\quad R := (\xp E-A)^{-1}B. \]

We're interested in the \emph{block-Krylov sequence}
\[
	R,\A R,\A^2R,\ldots
\]
\end{frame}




\begin{frame}{Moments of the transfer function about $\xp $}%
 
  Via Neumann (geometric series) expansion, 
  		\begin{align*}
  			H(s) &= B^T\left(I-(s-\xp )\A\right)^{-1}R \\
  				 &= B^T\left(\sum_{j=0}^\infty (s-\xp)^j\A^j\right)R\\
  				 &= \sum_{j=0}^\infty (s-\xp)^j \Balert{B^T\A^j R}
  		\end{align*}     		
  \begin{itemize}
  		\item This the Taylor series expansion of $H(s)$ about $\xp$.
  		\item Block-Krylov sequence
  		      \[ R,\A R,\A^2R,\ldots \Balert{\A^j R},\ldots\]
  \end{itemize}   
\end{frame}
 
 
 
 
\begin{frame}{Moment matching}{}
...suggests \alert{$n$-th Block-Krylov subspace}
\[ 
  \krylov{n}{\A}{R}:= \colspan\left\{R,\A R,\A^2R,\ldots,\A^{n-1}R\right\}.
\]

For $V\in\R^{N\times n}$ such that
 \[  \krylov{n}{\A}{R} \subseteq \text{range}\,V, \]
 	
ROM via projection on to $V$ matches $n$ moments about $\xp$.
\[
\Bn^T\widetilde{\A}^j \widetilde{R} = B^T\A^j R\quad\text{for}\quad j=0,1,2,\ldots,n-1
\]
\begin{itemize}
\item because $\Bn^T \widetilde{\A}^j \widetilde{R} 
= B^T \Balert{V \widetilde{\A}^j \widetilde{R}} = B^T\Balert{\A^j R }$  
\end{itemize}
\end{frame}



\begin{frame}{Progressive Discovery of Krylov Space}
For general $\xp\in\C$,
\[ \A := -(\xp E-A)^{-1}E\quad\text{and}\quad R := (\xp E-A)^{-1}B \]
\[ 
  \krylov{n}{\A}{R}:= \colspan\left\{R,\A R,\A^2R,\ldots,\A^{n-1}R\right\}
	\subset\C^N
\]

\medskip
$H:\krylov{j}{\A}{R}\rightarrow\krylov{j+1}{\A}{R}$ 
\emph{advances} the Krylov subspace.

\medskip
\begin{itemize}
\item Not Real! 

\item We need $V\in\R^{N\times n}$ for which 
\[ \krylov{n}{\A}{R} \subseteq \text{range}\,V. \]
\end{itemize}
\end{frame}

\begin{frame}{Non-Real Expansion Point?}
For $\xp\in\C$, $\A=\A(\xp)$ and $R=R(\xp)$ 

\medskip
Grimme (1997) suggested:
\begin{itemize}
\item Construct complex basis $V\in\C^{N\times\nblock}$ for
$\krylov{n}{\A}{R}$
\item Split $V$ into (separate $\text{Re}$ and $\text{Im}$ parts)
\[
  V^* := \mat{\re{v_1} & \im{v_1} & \re{v_2} & \im{v_2} & \cdots & \re{v_\nblock} & \im{v_\nblock}} 
  \in \R^{N \times 2\nblock}
 \]
\item Then, 
\begin{align*}
\colspan V^* &= \colspan \krylov{n}{\A}{R} \cup \krylov{n}{\conj{\A}}{\conj{R}}\\
&= \colspan \krylov{n}{\A(\xp)}{R(\xp)} \cup \krylov{n}{\A(\conj{\xp})}{R(\conj{\xp})}
\end{align*}
\end{itemize}
Match $n$ moments about points $\xp$ and $\conj{\xp}$ 
simultaneously .
\begin{itemize}
\alert{%
\item $\xp\notin\R \Rightarrow$ $2\times$ model size, $4\times$ computational cost
\item Re-orthogonalize $2\nblock$ vectors $v^*\in\R^N$?  AIEEE!
}
\end{itemize}
\end{frame}



\begin{frame}{Motivation for Complex Expansion Point(s)}{Example Pole Distribution}
Poles of transfer function $H(s) =  B^T(sE-A)^{-1}B $
correspond with eigenvalues of operator $\A$.
 \begin{figure}
  	    \centering
  	    \putfig[{735 420}]{0.8}{ex1841s1_poles_semilog.png}
  \end{figure}
  We want eigen-information associated with dominant poles. 
\end{frame}




\begin{frame}{Complex Split}

%\begin{definition}
 For a set of vectors $V\in\C^n$, the \emph{split} of $V$ is  
 \[
 V^* :=  \re{V} \,\cup\, \im{V} = \Set{ \text{Re}\, v \,\cup\, \text{Im}\, v | v\in V}\subset\R^N.
\]

 Similarly for a vector space $S = \colspan{V}$, 
 \begin{align*}
 	S^* &:= \colspan{V^*} \\
 	&\ = \colspan{\re{V} \,\cup\, \im{V}} \\
 	&\ = \colspan{V \cup \conj{V}\,}
\end{align*}
(spans over $\C$). 
%\end{definition}
\end{frame}


\begin{frame}{PCC-Krylov Subspace}
\begin{definition}
\emph{Paired Complex Conjugate (PCC)-Krylov}  subspace induced 
by $\A\in\C^{N\times N}$ and $R\in\C^N$.
 	\begin{align*}
 	\krylov{n}{\A}{R}^*
 	&= \colspan\krylov{n}{\A}{R} \ \cup\ \krylov{n}{\conj{\A}}{\conj{R}}\\
 	&= \colspan\Set{R, \conj{R}, \A R, \conj{\A R}, \A^2R, \conj{\A^2R},
 		\ldots,\A^{n-1}R, \conj{\A^{n-1}R} }\\
% 	&\subset R^N
 	\end{align*}
\end{definition}
\end{frame}



 \begin{frame}{Equivalent Real Formulation}
\begin{definition}
\emph{Realification} functor on $\C^N$ (Arnol'd,1992). 
\[ \lsup{\alert{\R}}{\C^N}=\R^{2N}\]  
 \[
 	x+iy \in \C^N \quad\longleftrightarrow\quad \mat{x\\y}\in\R^{2N} 
\]
\end{definition}

\medskip
For general  $\{v_1,v_2,\ldots,v_n\}$,
\[
	\lsup{\R}{\Set{\sum a_jv_j | a_j\in\C}}= \Set{\sum a_jv_j | a_j\in\R}.
\]
\begin{itemize}
\item Similarly, \emph{complexification} $\lsup{\C}{(\lsup{\R}{\C^N})}=\C^{N}$
\end{itemize}
\end{frame}



\begin{frame}{Equivalent Real Formulation}
For operator matrix $\A\in\C^{N\times N}$ and vector(s) $v\in\C^N$, 
 \[
  \Aeq = \mat{\re{\A} & -\im{\A}\\ \im{\A} & \re{\A} }\in\R^{2N\times 2N} 
  \quad\text{and}\quad \veq = \mat{\re{v}\\\im{v}}\in\R^{2N}.
 \]
  
 \begin{center} 
 $w = \mat{w^\text{\bf{t}}  \\ w^\text{\bf{b}}}\in\lsup{\R}{\C^N}$
 has \emph{top} and \emph{bottom} parts in $\R^N$.
\end{center}

Define the split of $W\subseteq\lsup{\R}{\C^N}$ as
 \[
 W^* :=  \tp{W} \,\cup\, \bt{W} = \Set{ \tp{w} \,\cup\, \bt{w} | w\in W}\subset\R^N.
\]
\end{frame}



\begin{frame}{Realified Krylov Subspace}{Cheaper Basis Computation}
 Recall the Krylov subspace  (over $\C$)
 \[ 
   \krylov{n}{\A}{R}:= \colspan\left\{R,\A R,\A^2R,\ldots,\A^{n-1}R\right\}
 	\subset\C^N.
 \]
 
\bigskip
Operator $\Aeq$, initial vector
 block $\Req$ yield (over $\R$)
 \[ 
    \krylov{n}{\Aeq}{\Req}:= 
    \colspan\left\{\Req,\Aeq \Req,\Aeq^2\Req,\ldots,\Aeq^{n-1}\Req\right\}
  	\subset\R^{2N}.
 \]
 
 \bigskip
 Computing basis $\Veq$ for $\krylov{n}{\Aeq}{\Req}$ cuts cost in half.
 \begin{itemize}
 \item Why? Inner product in $\R^{2N}$ vs. $\C^N$
 
 \end{itemize}
 \end{frame}


\begin{frame}{Non-Equivalent Bases Yield The Same Split}
Bases $V$ and $\Veq$ of these isomorphic Krylov subspaces
\begin{align*}
    \colspan V &= \krylov{n}{\A}{R} \subset\C^N\\
 	\colspan \Veq &= \krylov{n}{\Aeq}{\Req}\subset\lsup{\R}{\C^N}
 \end{align*}
 are not equivalent:  $\lsup{\R}{V}\neq \Veq$ in general, but
 their splits 
 \[
 \colspan V^* = \colspan \Veq^* = \krylov{n}{\A}{R}^*\subset\R^N
 \]
 both span the same PCC-Krylov subspace.
 \begin{itemize}
 \item This follows from \[\lsup{\R}{(\cdot)}:\A^jR \leftrightarrow \Aeq^j\Req.\]
 \end{itemize}
 \end{frame}
 
 %\begin{comment}
 \begin{frame}{Eigeninformation in Realified Space}
 Recall equivalent operators 
 \[
 \A \in\C^{N\times N} 
 \quad\text{and}\quad  
 \Aeq = \mat{\re{\A} & -\im{\A}\\ \im{\A} & \re{\A} }\in\R^{2N\times 2N}. 
 \]

\medskip 
Spectra of $\A$ and $\Aeq = \lsup{\R}{H}$ related by
\[
\sigma(\Aeq) = \sigma(\A) \cup \conj{\sigma(\A)}
\]
but eigenspaces not equivalent.\\ For $\A v = \lambda v$,
\begin{align*}
\lsup{\R}{(\A v)} &= \lsup{\R}{(\lambda v)}\\
\Aeq\veq &= \mat{\re{\lambda}I &-\im{\lambda}I\\ \im{\lambda}I & \re{\lambda}I}\veq
\end{align*}
 \end{frame}
%\end{comment}


 \begin{frame}{Concluding Remarks}
 \begin{itemize}
 \item PCC-Krylov subspace is a complex space containing two 
 complex conjugate Krylov subspaces and admits a real basis.
 \item Realified Krylov subspaces are cheaper to work with and ideal
 for methods implementing the general complex expansion point.
 \end{itemize}
 \end{frame}

%\nocite{grimme97}
%\bibliographystyle{plain} 
%\bibliography{erensi_refs}

\end{document}


