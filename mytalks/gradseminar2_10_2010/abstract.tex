\documentclass{article}

\title{Complex expansion points and the PCC-Krylov subspace for projection-based model reduction}

\begin{document}
\maketitle
\begin{abstract}
Krylov subspace projection methods have been used for several years for
dimension reduction of large-scale models of input-output behavior
determined by a certain dynamical system.  In general they involve
forming
a Taylor approximation about an expansion point located somewhere on the
complex plane, but whose ideal location is unknown. Due to the increased
computation and storage required to perform arithmetic with complex
matrices, the use of complex expansion points has been of theoretical
interest but not used much in practice.

I will introduce a new variant of the standard Krylov subspace, called
the Paired Conjugate (PCC)-Krylov subspace, which is inevitable if we
are to consider general complex expansion points.  I will also introduce
the seemingly unrelated notion of equivalent-real arithmetic for complex
matrices and explain how it provides the key to making the use of these
subspaces feasible for industrial use.

This talk should be accessible to anyone with some background in complex
arithmetic and linear algebra, and knows what a Taylor series
approximation is.
\end{abstract}
\end{document}