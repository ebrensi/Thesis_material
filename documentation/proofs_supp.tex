% !TeX root = doc.tex


\section{Proofs and other supplementary information}

\subsection{Equivalent real Arnoldi is cheaper than complex Arnoldi}
\label{sec:eqrealcheaper}
When speaking of a complex vector $x+iy$ ($x,y\in\R$) and its real-equivalent form,
we make the identification
\[
	 x+iy \in \C^N \quad\leftrightarrow\quad \mat{x\\y}\in\R^{2N}.
\]
Observe that for two complex vectors $x+iy$ and  $a+ib$,
their euclidean inner product is computed as
\begin{equation}
	(x+iy)^H(a+ib) = x^T a + y^T b + i(x^T b - y^T a) \quad \in \C,
\label{eq:eucl_ip}
\end{equation}
which requires 4 real SAXPYs. The inner product of the same
vectors two in equivalent-real form is computed as
\begin{equation}
  	\mat{x\\y}^T \mat{a\\b} = x^T a + y^T b \quad \in \R,
\label{eq:split_ip}
\end{equation}
and uses 2 SAXPYs. Computing a complex basis in equivalent-real form via
Gram-Schmidt procedure requires only half the operations of computing
a basis using complex arithmetic.

The inner products \eqref{eq:eucl_ip} and \eqref{eq:split_ip} yield
different notions of orthogonality of a complex basis
and its equivalent-real counterpart, and ultimately incompatible spaces.  The inner product
\eqref{eq:split_ip} implies a weaker orthogonality than \eqref{eq:eucl_ip}:
if two complex vectors $v,w\in\C^N$ are orthogonal then it follows that their
equivalent real forms $\hat{v},\hat{w}\in\R^{2N}$ are also orthogonal, but the converse
is not true in general.
A basis $\Veq$ of the block-Krylov subspace $\krylov{n}{\Heq}{\Req}$ cannot be
identified with a basis of $\krylov{n}{\H}{\RR}$:  if
we express each basis vector $\veq_j$ as a complex vector
\[
v_j = \tp{\veq}^{(j)} + i\bt{\veq}^{(j)},
\]
the resulting set of complex vectors $\{v_j\}$ will generally neither be orthogonal,
nor will it span $\krylov{n}{\H}{\RR}$.  However, for double shifting we are not interested
in $\krylov{n}{\H}{\RR}$, but rather its PCC space $\krylov{n}{\H}{\RR}^*$,
which is why the results of Sec.~\ref{sec:equivbases} are remarkable.

The norms implied
by the inner product for a complex vector $v\in\C^N$ and its equivalent real form $\veq\in\R^{2N}$ are equal:
\[
\nrm{}{\veq} = \nrm{}{v}\in\R.
\]



\subsection{Equivalence of PCC spaces obtained via complex and equivalent-real formulation}
\label{sec:equivbases}
\begin{lemma}\label{lem:eqspaces}
Given $\H\in \C^{N\times N}$, $R\in\C^{N\times \nin}$, we denote the equivalent-real
formulations of $\H$ and $R$ as $\Heq\in\R^{2N \times 2N}$ and $\Req\in\R^{2N\times\nin}$.  Then
the PCC-Krylov subspaces induced by each pair are the same.
	\begin{equation}
	   	\krylov{n}{\Heq}{\Req}^* = \krylov{n}{\H}{\RR}^*
	\end{equation}
\end{lemma}

\begin{proof}
It is sufficient to show that $\Heq^j \Req$ is the
equivalent real formulation of $\H^j R$ for any integer $j\geq 0$.
Trivially for $j=0$ we have $\Req := \mat{\Re(R) \\ \Im(R)}$.
For $j\geq 1$, let $K = \H^{j-1} R$. Then $\Keq = \mat{\Re(K)\\ \Im(K)}$
is the equivalent-real form of $K = \Re(K) + i\Im(K)$, so
 \begin{equation*}
 \Heq^j \Req  = \Heq\Keq = \mat{\Re(\H) & -\Im(\H)\\ \Im(\H) & \Re(\H) }\mat{\Re(K) \\ \Im(K)}
 = \mat{\Re(\H)\Re(K) - \Im(\H)\Im(K) \\ \Re(\H)\Im(K) + \Im(\H)\Re(K) }
 \end{equation*}
is the equivalent real formulation of

\begin{equation*}
\H^j R = \H K = (\Re(\H) + i \Im(\H))(\Re(K) + i\Im(K))
 = \left( \Re(\H)\Re(K) - \Im(\H)\Im(K) \right) +
	i\left(\Re(\H)\Im(K) + \Im(\H)\Re(K)  \right)
\end{equation*}

\end{proof}


\bigskip

\bigskip
The next results establish that the basis vectors produced by an iteration
of the Arnoldi process advance the PCC-Krylov subspace in the same order,
regardless of whether we use complex or equivalent-real formulation.
We first show this for the case that $R=r$ is a single vector ($p=1$), and
it follows for the general block case ($p\geq 1$) via corollary.


\begin{theorem} \label{thm:eqbases}Take $\H$, $r$ and their equivalent-real
formulations $\Heq$ and $\req$.  Let
$V_n$ be the ordered set of vectors output from $n$ Arnoldi iterations of $\H$
on start vector $r$,
and let $\Veq_n$ be the vectors produced by Arnoldi iterations using $\Heq$ and $\req$.
Then there exist real scalars $\alpha, \beta$ and real
$w,z\in \spn{\V_n}=\krylov{n}{\H}{r}^*$ such that
	\begin{equation}
	   	\Re(v_n) = \alpha\tp{\veq}^{(n)} + w \qquad\text{and }\qquad
		 \Im(v_n) = \beta\bt{\veq}^{(n)} + z,
     \label{eq:equ_bases}
	\end{equation}
where $w_0=z_0=0$.
	
\end{theorem}

\begin{proof}
For $n=1$ we have $v_1 = r/\nrm{}{r}$, and $\veq_1 = \req / \nrm{}{r}$, so $\Re(v_1) = \tp{\veq}^{(1)}$ and
$\Im(v_1) = \bt{\veq}^{(1)}$, trivially satisfying \eqref{eq:equ_bases}.

Noting that $\spn{V_n}=\krylov{n}{\H}{r}\subseteq \krylov{n}{\H}{r}^*$, for $n\geq 1$,
the Arnoldi process computes scalars $h_{n+1,n}\in\R$ and $\{h_{ij}\}\subset\C$ such
that
	\begin{align*}
	h_{n+1,n}v_{n+1} &= \H v_{n} - \sum_{j=1}^{n} h_{jn} v_j \\
	 &= \H^{n}r + \sum_{j=1}^{n} c_j v_j, \qquad c_j\in\C\\
	 &= \H^{n}r + \sum_{j=1}^{\nblocksplit} d_j\vt_j, \qquad d_j\in\C\\
	 &= \H^{n}r + w_1 + iz_1, \qquad w_1,z_1\in\krylov{n}{\H}{r}^* \cap \R^N.
	\end{align*}
Applying Lemma~\ref{lem:eqspaces}, in equivalent real
form:
\begin{equation}
h_{n+1,n}\mat{\Re(v_{n+1})\\ \Im(v_{n+1}}) = \Heq^{n}\req + \mat{w_1\\z_1}
\label{eq:cplx_prog}
\end{equation}

Similarly,
$n$ Arnoldi iterations with $\Heq$ and $\req$ yield
\begin{equation}
\heq_{n+1,n}\mat{\tp{\veq}^{(n+1)}\\ \bt{\veq}^{(n+1)}} =
\heq_{n+1,n}\veq_{n} =
 \Heq^{n}\req + \sum_{j=1}^{n} \hat{c}_j\veq_j
= \Heq^{n}\req + \sum_{j=1}^{\nblocksplit} \mat{\hat{a}_j\vt_j \\ \hat{b}_j\vt_j}
= \Heq^{n}\req + \mat{w_2 \\ z_2},
\label{eq:eqreal_prog}
\end{equation}
where $\hat{c}_j\in\C$, $\hat{a}_j,\hat{b}_j\in\R$, and $w_2,z_2\in\krylov{n}{\H}{r}^* \cap \R^N$.

\smallskip
Then \eqref{eq:equ_bases} follows from
\eqref{eq:cplx_prog} and \eqref{eq:eqreal_prog}.
\end{proof}

Theorem~\ref{thm:eqbases} establishes that Arnoldi vectors generated using $\Heq$ and $\req$
yield basis vectors for $\krylov{n}{\H}{r}^*$ in the same order as those obtained from $\H$ and $r$;
in fact, up to finite precision error they yield exactly the same basis.

\begin{corollary}
Something analagous to the result of Theorem~\ref{thm:eqbases}, except for a start block $R$ instead
of a single vector $r$.  Not sure how to state this.
\end{corollary}

\begin{proof}
Proof here.
\end{proof}


\subsection{Ritz vectors from Equivalent Real Arnoldi process}
As discussed in Section~\ref{sec:eqreal}, a $n$ iterations of the Band-Arnoldi process with
$\Heq\in\R^{2N \times 2N}$ and starting block $\Req\in\R^{2N \times \nin}$ yield
the basis $\Veq_n\in\R^{2N \times \nblockeq}$ of the Krylov subspace $\krylov{n}{\Heq}{\Req}$,
and the Arnoldi matrix $\Heq_n=\Veq_n^T \Heq \Veq_n\in\R^{\nblockeq \times \nblockeq}$.
We illustrate two ways to extract information about convergence of the ROM implied thus far,
via  approximate poles.

\subsubsection{Via explicit projection}
Explicit projection in general was introduced in Section~\ref{sec:explicitprojection}.
Here, we re-hash it in more detail and and with respect
to the realified (equivalent-real) formulations of section~\ref{sec:makingrealV}.
A ROM obtained via explicit projection is simply the original matrices $A$,$E$, and $B$
determining the unreduced model, projected on to a suitable subspace of $\R^N$,
which in our case is the PCC-Krylov subspace $\krylov{n}{\H}{\RR}^*$.
From $\Veq_n$ we obtain $\V_n\in\R^{N \times \nblocksplit}$, an orthonormal basis for the PCC-Krylov
subspace $\krylov{n}{\H}{\RR}^*$ (see Section~\ref{sec:eqreal}, and note
that $n\leq\nblocksplit\leq 2n\nin$).


\begin{equation*}
          A_n := \V_n^TA\V_n, \quad E_n := \V_n^TE\V_n, \quad B_n := \V_n^TB.
  \label{}
\end{equation*}
The implied ROM transfer function has the familiar format
 \begin{equation*}
     H_n(s) = B_n^T\left(sE_n-A_n\right)^{-1}B_n.
    \tag{\ref{eq:rm_proj}}
 \end{equation*}
Poles of \eqref{eq:rm_proj} are values $\rpol\in\C$ such that $\nrm{}{H_n(\rpol)}=\infty$, i.e.
generalized eigenvalues of the matrix pencil $sE_n-A_n$.  These  poles approximate those
of the unreduced model transfer function, and thus convergence of these Ritz-values
(approximate eigenvalues) is an indicator of the quality of the ROM.
There are $\nblocksplit$ of these eigenvalues and they come in complex conjugate pairs,
yielding essentially $\nblocksplit / 2$ uniquely determined values.

An eigenpair $(\rpol,w)$ of $sE_n-A_n$, satisfies the \emph{Galerkin} condition
$\V_n^T(\rpol E - A)z=0$, for Ritz-vector $z=\V_n w$.
Our measure of convergence of $(\rpol,z)$ is the explicitly computed relative residual error
  \begin{equation*}
      \mathrm{rr_{proj}} = \frac{\nrm{}{\rpol Ez - Az}}{\nrm{}{Az}}.
  \end{equation*}


\begin{figure}[htbp]
	\centering
	\subfloat[]{\putfig{.48}{rrproj50_308s_1.png}}\hfill
	\subfloat[\label{fig:tfadj}]{\putfig{.48}{tfuncproj50_308s_1.png}}
	\caption{Relative residual error (determined via explicit projection)
	 after $n=50$ iterations on example data set
	 $\mathtt{ex308s_1}$ with $\xp=10^9\pi(.5+i)$.  Convergence tolerance of $10^{-5}$
	 suggests $7$ conjugate pairs of converged Ritz-values.  \subref{fig:tfadj}
	 is the corresponding transfer function, computed via \eqref{eq:rm_proj}.
 	}
	\label{fig:rrproj308s1w50}
\end{figure}



\subsubsection{Via eigenvalues of Arnoldi matrix}
For this method, we use both Arnoldi matrix $\Heq_n$ and $\Veq_n$.
Given that $\H$ is diagonalizable as  $\H = X\Lambda X^{-1}$ (we consider its
Jordan-normal form if not),  the equivalent-real formulation
$\Heq$ can be factored \cite[Proposition 5.1]{AN} as
\begin{equation}
Y^{-1}\Heq Y = \mat{\Lambda& 0\\0&\conj{\Lambda}} \quad\text{where}\quad
Y = \frac{1}{\sqrt{2}}\mat{X&-i\conj{X}\\-iX&\conj{X}}
.
\label{eq:eqrealeigstruct}
\end{equation}

This gives the following relationships between eigenvalues/vectors of $\H$
and $\Heq$:
\begin{itemize}
	\item[(i)] $\sigma(\Heq) = \sigma(\H) \cup \sigma(\conj{\H})$
	\item[(ii)] For an eigenpair $(\lambda,x)$ of $\H$, it follows that  $(\lambda,y)$ and
		$(\conj{\lambda},\conj{y})$ are eigenpairs of $\Heq$, where $y=\mat{x \\-ix}$.
	\item[(iii)] For an eigenpair $(\lambda,y)$ of $\Heq$, there exists a unique
	$x\in\C^N$ such that either
	\[  y = \mat{x\\-ix}, \quad \lambda \in\sigma(\H), \qquad\text{or}\qquad
		y = \mat{-i\conj{x}\\ \conj{x}},\quad  \lambda \in\sigma(\conj{\H}).
	\]
		and incidentally $x$ is an eigenvector of $\H$.
\end{itemize}
For simplicity of notation we have omitted the $1/\sqrt{2}$ factor, but it is necessary
to preserve the unit property of the eigenvectors,
i.e. $\nrm{2}{x}=1 \Leftrightarrow \nrm{2}{y}=1$.

The Arnoldi process with $\Heq$ and starting block $\Req$ provides us with
 Arnoldi (``Hessenberg'') matrix $\Heq_n$, and Arnoldi vectors $\Veq_n$.
Eigenvalue decomposition of $\Heq_n$ yields Ritz-values (approximate eigenvalues)
$\reig\in\sigma(\Heq_n)$
of $\Heq$ each of which is an approximate eigenvalue
either of $\H$ or of $\conj{\H}$,
and eigenspace $W$ such that $Z=\Veq_n W$ is an approximate eiegenspace of
$\Heq$.  We want to approximate $\sigma(\H)$, but it is not
immediately apparent if a Ritz-value $\reig$ is an approximate eigenvalue of $\H$ or of
$\conj{\H}$. Solution of the following minimization problem allows us to distinguish the
Ritz-values:
In light of \eqref{eq:eqrealeigstruct} we determine the approximate
eigenspace
\begin{equation}
\widetilde{X} = f(Z) := \argmin_X\left\| Z - \frac{1}{\sqrt{2}}\mat{X\\-iX} \right\|
= \frac{1}{\sqrt{2}}\left(\tp{Z}+i\bt{Z}\right)
\label{eq:minslvX}
\end{equation}
of $\H$. Next, we consider the \emph{structurally-correct} variation of $Z$,
\begin{equation}
 	Z_{\text{alt}} = g(Z) := \frac{1}{\sqrt{2}}\mat{\widetilde{X}\\-i\widetilde{X}}
 	 = \frac{1}{\sqrt{2}}\mat{f(Z)\\-if(Z)}
 	= \frac{1}{2}(Z+i\mat{\bt{Z}\\ -\tp{Z}})
 	.
 \label{eq:Zalt}
\end{equation}

\eqref{eq:minslvX} and \eqref{eq:Zalt} provide alternatives
to relative residual as a measure of Ritz-vector convergence, as well as a simple
way to determine whether a Ritz-value $\reig$ of $\Heq$ is associated with $\H$,
or with $\conj{\H}$. Our numerical experiments did not show $Z_{\text{alt}}$ to be a
better approximation than $Z$, in general.



\subsubsection{Convergence criteria for Arnoldi matrix eigenvalues}
Consider $f$, defined
in \eqref{eq:minslvX}.  For an exact eigen-pair $(\lambda, y)$ of $\Heq$
where $\lambda\in\sigma(\H)$, we have that
that $y=\mat{x&-ix}^T$, where $(\lambda,x)$ is the corresponding
exact eigen-pair of $\H$.
Then $f(y) = x$, and $f(\conj{y})=0$, or using \eqref{eq:Zalt}, we have $y_\text{alt}=g(y)=y$
and $g(\conj{y})=0$.
Thus, for a Ritz-pair $(\reig,z)$ of $\Heq$, we can expect that
\emph{form-error} $\nrm{2}{z-z_{\text{alt}}}\approx 0$
 only if $\reig$ is sufficiently converged, \emph{and} it is
an approximate eigenvalue of $\H$. Combining \eqref{eq:minslvX} and \eqref{eq:Zalt}
and assuming $\nrm{2}{z}=1$,
\begin{align*}
4\nrm{2}{z-z_\text{alt}}^2 &= \nrm{2}{\tp{z}-i\bt{z}}^2 + \nrm{2}{\tp{z}-i\bt{z}}^2 \\
&= (\tp{z}-i\bt{z})^H(\tp{z}-i\bt{z}) +(\tp{z}-i\bt{z})^H(\tp{z}-i\bt{z})\\
&= 2+4\operatorname{Im}(\tp{z}^H\bt{z}).
\end{align*}

 In summary, for $z_\text{alt}=g(z)$ and $\widetilde{x}=f(z)$,
 \begin{equation}
 0\quad\leq\quad \nrm{2}{z-z_\text{alt}}^2 		 \quad=\quad
 \frac{1}{2}+ \operatorname{Im}(\tp{z}^H\bt{z})  \quad=\quad
 1 - \nrm{2}{\widetilde{x}}^2 \quad\leq\quad 1
 \label{eq:formerr}
 \end{equation}
where
\[
\operatorname{Im}(\tp{z}^H\bt{z})=
\mat{-\operatorname{Im}{\tp{z}} \\ \operatorname{Re}{\tp{z}}}^T \mat{\operatorname{Re}{\bt{z}} \\ \operatorname{Im}{\bt{z}}},
\]
may present an ideal measure of convergence for Ritz-values of $\Heq$, and thus for poles of
the ROM transfer function determined via equivalent-real formulation.

 For a Ritz-pair $(\reig,z)$ of realified operator $\Heq$
($z=\Veq_n w$, where $(\reig,w)$ is an eigenpair of Arnoldi matrix $\Heq_n$),
we specify the relative-residual error, in general, as
\begin{equation*}
      \mathrm{rr} = \frac{\nrm{}{\Heq z-\reig z}}{\nrm{}{\reig z}}.
\end{equation*}
We compute two estimates of relative residual error:
\begin{itemize}
\item Via the Arnoldi relation, so that
$\Heq \Veq_n w = \reig \Veq_n w + \eta \veq_{n+1} e_n^{T} w$
implies
	 \begin{equation*}
	      \mathrm{rr_{arnoldi}} =  \frac{\nrm{2}{\Heq z-\reig z}}{\nrm{2}{\reig z}}
	      = \frac{\vert \eta(e_n^{T} w)\vert \nrm{2}{\veq_{n+1}}}{\vert \reig\vert\nrm{2}{z}}
	      = \left\vert\frac{\eta w_{n}}{\reig}\right\vert
      \end{equation*}

\item $\mathrm{rr_{explicit}}$, by explicitly computing $\Heq z$ and $\reig z$.  (This is
computationally impractical with typical applications, but we do it with our examples, for
comparison purposes.)
\end{itemize}
	
The infinity norm $\nrm{\infty}{z}=\max_j\|z_j \|$ can also be used and may be more efficient
to compute, but for the sake of simplicity we do not consider it here.

Different measures of convergence for the same Ritz-values are compared in
figure~\ref{fig:rrhess308s1w50}.
For every example that we have tried, the values indicate
roughly the same number of converged values/pairs for a given convergence tolerance.
The form-error for the eigenvalue represented in the $14$-th position
of figure~\ref{fig:rrhess308s1w50} is exactly $0$ (and is thus missing from the plot),
although relative residual errors indicate
that the Ritz/eigenvalue is not exact.  This suggests that it is possible for a Ritz-vector
to be of the correct form $\mat{x &-ix}^T$ given by \eqref{eq:eqrealeigstruct}, and yet
not be an eigenvector of $\Heq$,  i.e. the converse of \cite[Proposition 5.1]{AN}
does not hold.
	
\begin{figure}[htbp]
	\centering
	\putfig{.70}{rrhess50_308s_1.png}
	\caption{Relative residual error (determined via Arnoldi relation and explicitly computed)
	 and form-error squared $\nrm{2}{z-z_\textrm{alt}}^2$
	 of eigenvalues of $\Heq$ after $50$ iterations on example data set
	 $\mathtt{ex308s_1}$ with $\xp=10^9\pi(.5+i)$.
	 %Also, an alternate form error \texttt{relZdiff} given by $\nrm{\infty}{z-z_\textrm{alt}} / \nrm{\infty}{z}$.
	 Relative residual errors do not distinguish Ritz values
	 $\lambda$ and $\conj{\lambda}$ of $\Heq$, but form error does. Using the convergence tolerance
	 $10^{-5}$, residual error formulation determines $11$ converged eigenvalues (in conjugate pairs),
	 so that form-error is directly comparable to relative residual.
	 The form-error for the $14$-th position is not plotted because it
	 was computed to be exactly $0$.
	%The below figure is identical except that form error is not squared ($\nrm{2}{z-z_\textrm{alt}}$).
	}
	\label{fig:rrhess308s1w50}
\end{figure}

\subsubsection{Equivalent-real `single-matrix' formulation of transfer function }
The transfer function of the unreduced model is expressed in single matrix formulation
(section~\ref{sec:singlematrix}) and as a Taylor expansion (section~\ref{sec:pade})
\begin{align}
     H(s) &= B^T\left(I-(s-\xp)\H\right)^{-1}R \tag{\ref{eq:tfunc_single_matrix}}\\
     &= \sum_{j=0}^\infty (s-\xp )^jB^T\H^jR \tag{\ref{eq:neumann}}.
\end{align}

For equivalent-real forms $\Heq$ and $\Req$, we have
\[
	\Heq^j \Req = \textrm{eqreal}(\H^j R)
	:= \mat{\operatorname{Re}\H^j R \\ \operatorname{Im}\H^j R}
\]
for integer $j\geq 0$
(see proof of Lemma~\ref{lem:eqspaces} in section~\ref{sec:equivbases}),
so for $\Beq:=\mat{B\\-iB}$ it follows that $\Beq^H\Heq^j \Req = B^T\H^j R$.
Thus, the transfer function is defined equivalently as
\begin{equation}
     H(s) = \Beq^H\left(I-(s-\xp)\Heq\right)^{-1}\Req
   \label{eq:eq_real_tfunc_single_matrix}
\end{equation}
(here, the identity $I\in\R^{2N \times 2N}$).

A ROM  matching $n$-moments about $\xp$, has
\begin{align*}
   H(s) &= \sum_{j=0}^n (s-\xp )^jB^T\H^jR + \bigO{(s-\xp )^{n+1}}\\
   &= \sum_{j=0}^n (s-\xp )^j\Beq^T\Heq^j\Req + \bigO{(s-\xp )^{n+1}}.
\end{align*}
